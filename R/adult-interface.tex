% Options for packages loaded elsewhere
\PassOptionsToPackage{unicode}{hyperref}
\PassOptionsToPackage{hyphens}{url}
%
\documentclass[
]{article}
\usepackage{amsmath,amssymb}
\usepackage{iftex}
\ifPDFTeX
  \usepackage[T1]{fontenc}
  \usepackage[utf8]{inputenc}
  \usepackage{textcomp} % provide euro and other symbols
\else % if luatex or xetex
  \usepackage{unicode-math} % this also loads fontspec
  \defaultfontfeatures{Scale=MatchLowercase}
  \defaultfontfeatures[\rmfamily]{Ligatures=TeX,Scale=1}
\fi
\usepackage{lmodern}
\ifPDFTeX\else
  % xetex/luatex font selection
\fi
% Use upquote if available, for straight quotes in verbatim environments
\IfFileExists{upquote.sty}{\usepackage{upquote}}{}
\IfFileExists{microtype.sty}{% use microtype if available
  \usepackage[]{microtype}
  \UseMicrotypeSet[protrusion]{basicmath} % disable protrusion for tt fonts
}{}
\makeatletter
\@ifundefined{KOMAClassName}{% if non-KOMA class
  \IfFileExists{parskip.sty}{%
    \usepackage{parskip}
  }{% else
    \setlength{\parindent}{0pt}
    \setlength{\parskip}{6pt plus 2pt minus 1pt}}
}{% if KOMA class
  \KOMAoptions{parskip=half}}
\makeatother
\usepackage{xcolor}
\usepackage[margin=1in]{geometry}
\usepackage{color}
\usepackage{fancyvrb}
\newcommand{\VerbBar}{|}
\newcommand{\VERB}{\Verb[commandchars=\\\{\}]}
\DefineVerbatimEnvironment{Highlighting}{Verbatim}{commandchars=\\\{\}}
% Add ',fontsize=\small' for more characters per line
\usepackage{framed}
\definecolor{shadecolor}{RGB}{248,248,248}
\newenvironment{Shaded}{\begin{snugshade}}{\end{snugshade}}
\newcommand{\AlertTok}[1]{\textcolor[rgb]{0.94,0.16,0.16}{#1}}
\newcommand{\AnnotationTok}[1]{\textcolor[rgb]{0.56,0.35,0.01}{\textbf{\textit{#1}}}}
\newcommand{\AttributeTok}[1]{\textcolor[rgb]{0.13,0.29,0.53}{#1}}
\newcommand{\BaseNTok}[1]{\textcolor[rgb]{0.00,0.00,0.81}{#1}}
\newcommand{\BuiltInTok}[1]{#1}
\newcommand{\CharTok}[1]{\textcolor[rgb]{0.31,0.60,0.02}{#1}}
\newcommand{\CommentTok}[1]{\textcolor[rgb]{0.56,0.35,0.01}{\textit{#1}}}
\newcommand{\CommentVarTok}[1]{\textcolor[rgb]{0.56,0.35,0.01}{\textbf{\textit{#1}}}}
\newcommand{\ConstantTok}[1]{\textcolor[rgb]{0.56,0.35,0.01}{#1}}
\newcommand{\ControlFlowTok}[1]{\textcolor[rgb]{0.13,0.29,0.53}{\textbf{#1}}}
\newcommand{\DataTypeTok}[1]{\textcolor[rgb]{0.13,0.29,0.53}{#1}}
\newcommand{\DecValTok}[1]{\textcolor[rgb]{0.00,0.00,0.81}{#1}}
\newcommand{\DocumentationTok}[1]{\textcolor[rgb]{0.56,0.35,0.01}{\textbf{\textit{#1}}}}
\newcommand{\ErrorTok}[1]{\textcolor[rgb]{0.64,0.00,0.00}{\textbf{#1}}}
\newcommand{\ExtensionTok}[1]{#1}
\newcommand{\FloatTok}[1]{\textcolor[rgb]{0.00,0.00,0.81}{#1}}
\newcommand{\FunctionTok}[1]{\textcolor[rgb]{0.13,0.29,0.53}{\textbf{#1}}}
\newcommand{\ImportTok}[1]{#1}
\newcommand{\InformationTok}[1]{\textcolor[rgb]{0.56,0.35,0.01}{\textbf{\textit{#1}}}}
\newcommand{\KeywordTok}[1]{\textcolor[rgb]{0.13,0.29,0.53}{\textbf{#1}}}
\newcommand{\NormalTok}[1]{#1}
\newcommand{\OperatorTok}[1]{\textcolor[rgb]{0.81,0.36,0.00}{\textbf{#1}}}
\newcommand{\OtherTok}[1]{\textcolor[rgb]{0.56,0.35,0.01}{#1}}
\newcommand{\PreprocessorTok}[1]{\textcolor[rgb]{0.56,0.35,0.01}{\textit{#1}}}
\newcommand{\RegionMarkerTok}[1]{#1}
\newcommand{\SpecialCharTok}[1]{\textcolor[rgb]{0.81,0.36,0.00}{\textbf{#1}}}
\newcommand{\SpecialStringTok}[1]{\textcolor[rgb]{0.31,0.60,0.02}{#1}}
\newcommand{\StringTok}[1]{\textcolor[rgb]{0.31,0.60,0.02}{#1}}
\newcommand{\VariableTok}[1]{\textcolor[rgb]{0.00,0.00,0.00}{#1}}
\newcommand{\VerbatimStringTok}[1]{\textcolor[rgb]{0.31,0.60,0.02}{#1}}
\newcommand{\WarningTok}[1]{\textcolor[rgb]{0.56,0.35,0.01}{\textbf{\textit{#1}}}}
\usepackage{graphicx}
\makeatletter
\def\maxwidth{\ifdim\Gin@nat@width>\linewidth\linewidth\else\Gin@nat@width\fi}
\def\maxheight{\ifdim\Gin@nat@height>\textheight\textheight\else\Gin@nat@height\fi}
\makeatother
% Scale images if necessary, so that they will not overflow the page
% margins by default, and it is still possible to overwrite the defaults
% using explicit options in \includegraphics[width, height, ...]{}
\setkeys{Gin}{width=\maxwidth,height=\maxheight,keepaspectratio}
% Set default figure placement to htbp
\makeatletter
\def\fps@figure{htbp}
\makeatother
\setlength{\emergencystretch}{3em} % prevent overfull lines
\providecommand{\tightlist}{%
  \setlength{\itemsep}{0pt}\setlength{\parskip}{0pt}}
\setcounter{secnumdepth}{-\maxdimen} % remove section numbering
\ifLuaTeX
  \usepackage{selnolig}  % disable illegal ligatures
\fi
\IfFileExists{bookmark.sty}{\usepackage{bookmark}}{\usepackage{hyperref}}
\IfFileExists{xurl.sty}{\usepackage{xurl}}{} % add URL line breaks if available
\urlstyle{same}
\hypersetup{
  pdftitle={adult-interface.R},
  pdfauthor={doree},
  hidelinks,
  pdfcreator={LaTeX via pandoc}}

\title{adult-interface.R}
\author{doree}
\date{2024-07-24}

\begin{document}
\maketitle

\begin{Shaded}
\begin{Highlighting}[]
\CommentTok{\# generic methods for adult component}
\end{Highlighting}
\end{Shaded}

@title Derivatives for adult mosquitoes @description This method
dispatches on the type of \texttt{pars\$MYZpar}. @param t current
simulation time @param y state vector @param pars an \texttt{xds} object
@param s the species index @return the derivatives a {[}vector{]}
@export

\begin{Shaded}
\begin{Highlighting}[]
\NormalTok{dMYZdt }\OtherTok{\textless{}{-}} \ControlFlowTok{function}\NormalTok{(t, y, pars, s) \{}
  \FunctionTok{UseMethod}\NormalTok{(}\StringTok{"dMYZdt"}\NormalTok{, pars}\SpecialCharTok{$}\NormalTok{MYZpar[[s]])}
\NormalTok{\}}
\end{Highlighting}
\end{Shaded}

@title Compute the steady states as a function of the daily EIR
@description This method dispatches on the type of \texttt{MYZpar}.
@param Lambda the daily emergence rate of adult mosquitoes @param kappa
net infectiousness @param MYZpar a list that defines an adult model
@return none @export

\begin{Shaded}
\begin{Highlighting}[]
\NormalTok{xde\_steady\_state\_MYZ }\OtherTok{=} \ControlFlowTok{function}\NormalTok{(Lambda, kappa, MYZpar)\{}
  \FunctionTok{UseMethod}\NormalTok{(}\StringTok{"xde\_steady\_state\_MYZ"}\NormalTok{, MYZpar)}
\NormalTok{\}}
\end{Highlighting}
\end{Shaded}

@title Compute the steady states as a function of the daily EIR
@description This method dispatches on the type of \texttt{MYZpar}.
@param Lambda the daily emergence rate of adult mosquitoes @param MYZpar
a list that defines an adult model @return none @export

\begin{Shaded}
\begin{Highlighting}[]
\NormalTok{xde\_steady\_state\_M }\OtherTok{=} \ControlFlowTok{function}\NormalTok{(Lambda, MYZpar)\{}
  \FunctionTok{UseMethod}\NormalTok{(}\StringTok{"xde\_steady\_state\_M"}\NormalTok{, MYZpar)}
\NormalTok{\}}
\end{Highlighting}
\end{Shaded}

@title A function to set up adult mosquito models @description This
method dispatches on \texttt{MYZname}. @param MYZname the name of the
model @param pars a {[}list{]} @param s the species index @param EIPopts
is a {[}list{]} @param MYZopts a {[}list{]} @param calK is a
{[}matrix{]} @return {[}list{]} @export

\begin{Shaded}
\begin{Highlighting}[]
\NormalTok{xde\_setup\_MYZpar }\OtherTok{=} \ControlFlowTok{function}\NormalTok{(MYZname, pars, s, EIPopts, }\AttributeTok{MYZopts=}\FunctionTok{list}\NormalTok{(),  }\AttributeTok{calK=}\FunctionTok{diag}\NormalTok{(}\DecValTok{1}\NormalTok{))\{}
  \FunctionTok{class}\NormalTok{(MYZname) }\OtherTok{\textless{}{-}}\NormalTok{ MYZname}
  \FunctionTok{UseMethod}\NormalTok{(}\StringTok{"xde\_setup\_MYZpar"}\NormalTok{, MYZname)}
\NormalTok{\}}
\end{Highlighting}
\end{Shaded}

@title Derivatives for adult mosquitoes @description This method
dispatches on the type of \texttt{pars\$MYZpar}. @param t current
simulation time @param y state vector @param pars a {[}list{]} @param s
the species index @return the derivatives a {[}vector{]} @export

\begin{Shaded}
\begin{Highlighting}[]
\NormalTok{DT\_MYZt }\OtherTok{\textless{}{-}} \ControlFlowTok{function}\NormalTok{(t, y, pars, s) \{}
  \FunctionTok{UseMethod}\NormalTok{(}\StringTok{"DT\_MYZt"}\NormalTok{, pars}\SpecialCharTok{$}\NormalTok{MYZpar[[s]])}
\NormalTok{\}}
\end{Highlighting}
\end{Shaded}

@title Compute the steady states as a function of the daily EIR
@description This method dispatches on the type of \texttt{MYZpar}.
@inheritParams xde\_steady\_state\_MYZ @return none @export

\begin{Shaded}
\begin{Highlighting}[]
\NormalTok{dts\_steady\_state\_MYZ }\OtherTok{=} \ControlFlowTok{function}\NormalTok{(Lambda, kappa, MYZpar)\{}
  \FunctionTok{UseMethod}\NormalTok{(}\StringTok{"dts\_steady\_state\_MYZ"}\NormalTok{, MYZpar)}
\NormalTok{\}}
\end{Highlighting}
\end{Shaded}

@title A function to set up adult mosquito models @description This
method dispatches on \texttt{MYZname}. @param MYZname the name of the
model @param pars a {[}list{]} @param s the species index @param EIPopts
is a {[}list{]} @param MYZopts a {[}list{]} @param calK is a
{[}matrix{]} @return {[}list{]} @export

\begin{Shaded}
\begin{Highlighting}[]
\NormalTok{dts\_setup\_MYZpar }\OtherTok{=} \ControlFlowTok{function}\NormalTok{(MYZname, pars, s, EIPopts, }\AttributeTok{MYZopts=}\FunctionTok{list}\NormalTok{(),  }\AttributeTok{calK=}\FunctionTok{diag}\NormalTok{(}\DecValTok{1}\NormalTok{))\{}
  \FunctionTok{class}\NormalTok{(MYZname) }\OtherTok{\textless{}{-}}\NormalTok{ MYZname}
  \FunctionTok{UseMethod}\NormalTok{(}\StringTok{"dts\_setup\_MYZpar"}\NormalTok{, MYZname)}
\NormalTok{\}}
\end{Highlighting}
\end{Shaded}

@title Set bloodfeeding and mortality rates to baseline @description
This method dispatches on the type of \texttt{pars\$MYZpar}. It should
set the values of the bionomic parameters to baseline values. @param t
current simulation time @param y state vector @param pars a {[}list{]}
@param s the species index @return a {[}list{]} @export

\begin{Shaded}
\begin{Highlighting}[]
\NormalTok{MBionomics }\OtherTok{\textless{}{-}} \ControlFlowTok{function}\NormalTok{(t, y, pars, s) \{}
  \FunctionTok{UseMethod}\NormalTok{(}\StringTok{"MBionomics"}\NormalTok{, pars}\SpecialCharTok{$}\NormalTok{MYZpar[[s]])}
\NormalTok{\}}
\end{Highlighting}
\end{Shaded}

@title Time spent host seeking/feeding and resting/ovipositing
@description This method dispatches on the type of
\texttt{pars\$MYZpar}. @param t current simulation time @param pars a
{[}list{]} @return either a {[}numeric{]} vector if the model supports
this feature, or {[}NULL{]} @export

\begin{Shaded}
\begin{Highlighting}[]
\NormalTok{F\_tau }\OtherTok{\textless{}{-}} \ControlFlowTok{function}\NormalTok{(t, pars) \{}
  \FunctionTok{UseMethod}\NormalTok{(}\StringTok{"F\_tau"}\NormalTok{, pars}\SpecialCharTok{$}\NormalTok{MYZpar)}
\NormalTok{\}}
\end{Highlighting}
\end{Shaded}

@title Blood feeding rate of the infective mosquito population
@description This method dispatches on the type of
\texttt{pars\$MYZpar}. @param t current simulation time @param y state
vector @param pars a {[}list{]} @param s the species index @return a
{[}numeric{]} vector of length \texttt{nPatches} @export

\begin{Shaded}
\begin{Highlighting}[]
\NormalTok{F\_fqZ }\OtherTok{\textless{}{-}} \ControlFlowTok{function}\NormalTok{(t, y, pars, s) \{}
  \FunctionTok{UseMethod}\NormalTok{(}\StringTok{"F\_fqZ"}\NormalTok{, pars}\SpecialCharTok{$}\NormalTok{MYZpar[[s]])}
\NormalTok{\}}
\end{Highlighting}
\end{Shaded}

@title Blood feeding rate of the mosquito population @description This
method dispatches on the type of \texttt{pars\$MYZpar}. @param t current
simulation time @param y state vector @param pars a {[}list{]} @param s
the species index @return a {[}numeric{]} vector of length
\texttt{nPatches} @export

\begin{Shaded}
\begin{Highlighting}[]
\NormalTok{F\_fqM }\OtherTok{\textless{}{-}} \ControlFlowTok{function}\NormalTok{(t, y, pars, s) \{}
  \FunctionTok{UseMethod}\NormalTok{(}\StringTok{"F\_fqM"}\NormalTok{, pars}\SpecialCharTok{$}\NormalTok{MYZpar[[s]])}
\NormalTok{\}}
\end{Highlighting}
\end{Shaded}

@title Number of eggs laid by adult mosquitoes @description This method
dispatches on the type of \texttt{pars\$MYZpar}. @param t current
simulation time @param y state vector @param pars a {[}list{]} @param s
the species index @return a {[}numeric{]} vector of length
\texttt{nPatches} @export

\begin{Shaded}
\begin{Highlighting}[]
\NormalTok{F\_eggs }\OtherTok{\textless{}{-}} \ControlFlowTok{function}\NormalTok{(t, y, pars, s) \{}
  \FunctionTok{UseMethod}\NormalTok{(}\StringTok{"F\_eggs"}\NormalTok{, pars}\SpecialCharTok{$}\NormalTok{MYZpar[[s]])}
\NormalTok{\}}
\end{Highlighting}
\end{Shaded}

@title Return the variables as a list @description This method
dispatches on the type of \texttt{pars\$MYZpar{[}{[}s{]}{]}}. @param y
the variables @param pars a {[}list{]} @param s the vector species index
@return a {[}list{]} @export

\begin{Shaded}
\begin{Highlighting}[]
\NormalTok{list\_MYZvars }\OtherTok{\textless{}{-}} \ControlFlowTok{function}\NormalTok{(y, pars, s) \{}
  \FunctionTok{UseMethod}\NormalTok{(}\StringTok{"list\_MYZvars"}\NormalTok{, pars}\SpecialCharTok{$}\NormalTok{MYZpar[[s]])}
\NormalTok{\}}
\end{Highlighting}
\end{Shaded}

@title Put MYZvars in place of the MYZ variables in y @description This
method dispatches on the type of \texttt{pars\$MYZpar{[}{[}s{]}{]}}.
@param MYZvars the variables @param y the variables @param pars a
{[}list{]} @param s the vector species index @return a {[}list{]}
@export

\begin{Shaded}
\begin{Highlighting}[]
\NormalTok{put\_MYZvars }\OtherTok{\textless{}{-}} \ControlFlowTok{function}\NormalTok{(MYZvars, y, pars, s) \{}
  \FunctionTok{UseMethod}\NormalTok{(}\StringTok{"put\_MYZvars"}\NormalTok{, pars}\SpecialCharTok{$}\NormalTok{MYZpar[[s]])}
\NormalTok{\}}
\end{Highlighting}
\end{Shaded}

@title A function to set up adult mosquito models @description This
method dispatches on \texttt{MYZname}. @param pars a {[}list{]} @param s
the species index @param MYZopts a {[}list{]} @return {[}list{]} @export

\begin{Shaded}
\begin{Highlighting}[]
\NormalTok{setup\_MYZinits }\OtherTok{=} \ControlFlowTok{function}\NormalTok{(pars, s, }\AttributeTok{MYZopts=}\FunctionTok{list}\NormalTok{())\{}
  \FunctionTok{UseMethod}\NormalTok{(}\StringTok{"setup\_MYZinits"}\NormalTok{, pars}\SpecialCharTok{$}\NormalTok{MYZpar[[s]])}
\NormalTok{\}}
\end{Highlighting}
\end{Shaded}

@title Add indices for adult mosquitoes to parameter list @description
This method dispatches on the type of \texttt{pars\$MYZpar}. @param pars
a {[}list{]} @param s the species index @return {[}list{]} @export

\begin{Shaded}
\begin{Highlighting}[]
\NormalTok{make\_indices\_MYZ }\OtherTok{\textless{}{-}} \ControlFlowTok{function}\NormalTok{(pars, s) \{}
  \FunctionTok{UseMethod}\NormalTok{(}\StringTok{"make\_indices\_MYZ"}\NormalTok{, pars}\SpecialCharTok{$}\NormalTok{MYZpar[[s]])}
\NormalTok{\}}
\end{Highlighting}
\end{Shaded}

@title Parse the outputs and return the variables by name in a list
@description This method dispatches on the type of
\texttt{pars\$MYZpar}. It computes the variables by name and returns a
named list. @param outputs a {[}matrix{]} of outputs from deSolve @param
pars a {[}list{]} that defines a model @param s the species index
@return {[}list{]} @export

\begin{Shaded}
\begin{Highlighting}[]
\NormalTok{parse\_outputs\_MYZ }\OtherTok{\textless{}{-}} \ControlFlowTok{function}\NormalTok{(outputs, pars, s) \{}
  \FunctionTok{UseMethod}\NormalTok{(}\StringTok{"parse\_outputs\_MYZ"}\NormalTok{, pars}\SpecialCharTok{$}\NormalTok{MYZpar[[s]])}
\NormalTok{\}}
\end{Highlighting}
\end{Shaded}

@title Return initial values as a vector @description This method
dispatches on the type of \texttt{pars\$MYZpar}. @param pars a
{[}list{]} @param s the species index @return {[}numeric{]} @export

\begin{Shaded}
\begin{Highlighting}[]
\NormalTok{get\_inits\_MYZ }\OtherTok{\textless{}{-}} \ControlFlowTok{function}\NormalTok{(pars, s) \{}
  \FunctionTok{UseMethod}\NormalTok{(}\StringTok{"get\_inits\_MYZ"}\NormalTok{, pars}\SpecialCharTok{$}\NormalTok{MYZpar[[s]])}
\NormalTok{\}}
\end{Highlighting}
\end{Shaded}

@title Set the initial values as a vector @description This method
dispatches on the type of \texttt{pars\$MYZpar}. @param pars a
{[}list{]} @param y0 a vector of variable values from a simulation
@param s the species index @return a {[}list{]} @export

\begin{Shaded}
\begin{Highlighting}[]
\NormalTok{update\_inits\_MYZ }\OtherTok{\textless{}{-}} \ControlFlowTok{function}\NormalTok{(pars, y0, s) \{}
  \FunctionTok{UseMethod}\NormalTok{(}\StringTok{"update\_inits\_MYZ"}\NormalTok{, pars}\SpecialCharTok{$}\NormalTok{MYZpar[[s]])}
\NormalTok{\}}
\end{Highlighting}
\end{Shaded}

@title Convert a model from dde to the corresponding ode @description
This method dispatches on the type of \texttt{pars\$MYZpar\$xde} @param
pars a {[}list{]} @return a {[}list{]} @export

\begin{Shaded}
\begin{Highlighting}[]
\NormalTok{dde2ode\_MYZ }\OtherTok{=} \ControlFlowTok{function}\NormalTok{(pars)\{}
  \FunctionTok{UseMethod}\NormalTok{(}\StringTok{"dde2ode\_MYZ"}\NormalTok{, pars}\SpecialCharTok{$}\NormalTok{MYZpar}\SpecialCharTok{$}\NormalTok{xde)}
\NormalTok{\}}
\end{Highlighting}
\end{Shaded}

@title Convert a model from dde to the corresponding ode @description If
it is already an ode, return pars unchanged. @param pars a {[}list{]}
@return a {[}list{]} @export

\begin{Shaded}
\begin{Highlighting}[]
\NormalTok{dde2ode\_MYZ.ode }\OtherTok{=} \ControlFlowTok{function}\NormalTok{(pars)\{pars\}}
\end{Highlighting}
\end{Shaded}

@title Convert a model from dde to the corresponding ode @description If
it is a dde, return the corresponding ode @param pars a {[}list{]}
@return a {[}list{]} @export

\begin{Shaded}
\begin{Highlighting}[]
\NormalTok{dde2ode\_MYZ.dde }\OtherTok{=} \ControlFlowTok{function}\NormalTok{(pars)\{}
\NormalTok{  pars}\SpecialCharTok{$}\NormalTok{MYZpar}\SpecialCharTok{$}\NormalTok{xde }\OtherTok{\textless{}{-}} \StringTok{"ode"}
\NormalTok{  pars}\SpecialCharTok{$}\NormalTok{MYZpar}\SpecialCharTok{$}\NormalTok{solve\_as }\OtherTok{\textless{}{-}} \StringTok{"ode"}
\NormalTok{  pars }\OtherTok{\textless{}{-}} \FunctionTok{xde\_make\_MYZpar\_RM}\NormalTok{(pars, MYZopts}\OtherTok{\textless{}{-}}\NormalTok{ pars}\SpecialCharTok{$}\NormalTok{MYZpar,}
                             \AttributeTok{calK=}\NormalTok{pars}\SpecialCharTok{$}\NormalTok{MYZpar}\SpecialCharTok{$}\NormalTok{calK)}
\NormalTok{  pars }\OtherTok{\textless{}{-}} \FunctionTok{make\_indices}\NormalTok{(pars)}
  \FunctionTok{return}\NormalTok{(pars)}
\NormalTok{\}}
\end{Highlighting}
\end{Shaded}


\end{document}
